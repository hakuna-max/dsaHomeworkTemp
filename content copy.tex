%!TEX root = main.tex
% !TEX encoding = UTF-8
% content.tex
% 请在这里撰写你所有的作业内容

% --- 第一题 ---
\section{问题一:算法分析}

这里是问题一的详细描述。例如,请分析以下算法的时间复杂度。

\subsection*{解:}

在这里撰写你的解答。

\noindent \textbf{公式的编辑:}

你可以使用 \mintinline{latex}{$} $\cdots$ \mintinline{latex}{$} 来插入行内数学公式,例如 \mintinline{latex}{$O(n \log n)$},这会生成“$O(n \log n)$”。对于独立的公式,你可以使用 \mintinline{latex}{equation} 环境:

\begin{listing}[ht!]
    \begin{minted}{latex}
        \begin{equation}
            \label{eq: masterTheorem} % 公式的标签,方便在正文中使用 \ref{eq: masterTheorem} 引用该公式
            T(n) = 2T(n/2) + O(n)
        \end{equation}
    \end{minted}
    \caption{\mintinline{latex}{equation} 环境示例}
    \label{lst: equaiton-example}
\end{listing}

Listing~\ref{lst: equaiton-example} 展示了如何使用 \mintinline{latex}{equation} 环境来插入数学公式。该环境可以在文档中生成独立的数学公式,例如:

\begin{equation}
    T(n) = 2T(n/2) + O(n)
\end{equation}

\noindent \textbf{图片的插入:}

你可以使用 \mintinline{latex}{figure} 环境来插入图片。例如:
\begin{listing}[ht!]
    \begin{minted}{latex}
        \begin{figure}[ht!]
            \centering % 图片居中
            \includegraphics[width=\textwidth]{computation.pdf} % 假设所有图片放在了images文件夹下,图片文件名为 computation.pdf。width=\textwidth参数控制图片宽度为文本宽度,可以根据需要调整,比如 width=0.8\textwidth。
            \caption{示例图片}
            \label{fig: example-image} % 图片的标签,方便在正文中使用 \ref{fig: example-image} 引用该图片
        \end{figure}
    \end{minted}
    \caption{\mintinline{latex}{figure} 环境示例}
    \label{lst: figure-example}
\end{listing}

Listing~\ref{lst: figure-example} 展示了如何使用 \mintinline{latex}{figure} 环境来插入图片。该环境可以在文档中生成图片,例如:

\begin{figure}[ht!]
    \centering
    \includegraphics[width=\textwidth]{computation.pdf}
    \caption{示例图片}
    \label{fig: example-image}
\end{figure}

% --- 第二题 ---
\section{问题二:代码实现}

请使用 Python 实现一个链表的逆序操作。

\subsection*{解:}


以下是我的 Python 代码实现。

这里使用 \mintinline{latex}{minted} 宏包来展示代码,你需要确保你的 \LaTeX 编译器开启了 \mintinline{shell}{-shell-escape} 选项(假设您使用提供的模板,并使用VSCode作为编辑器,其中的\mintinline{text}{./vscode/settings.json}已添加好相关设置)。

\noindent \textbf{代码的编辑:}

代码块的编辑可以使用 \mintinline{latex}{minted} 环境。如Listing~\ref{lst: minted-example} 所示:

\begin{listing}[ht!]
    \begin{lstlisting}[
        language=TeX,
        numbers=left, 
        numberstyle=\tiny\color{black},
        numbersep=1em,
        xleftmargin=2em,
        lineskip=0.2\baselineskip
        ]
\begin{minted}{python}
    class ListNode:
        def __init__(self, x):
            self.val = x
            self.next = None
\end{minted}
    \end{lstlisting}
    \caption{minted 环境使用示例}
    \label{lst: minted-example}
\end{listing}

Listing~\ref{lst: minted-example} 展示了如何使用 \mintinline{latex}{minted} 环境来插入代码。该环境可以在文档中生成高亮的代码块,如Listing~\ref{lst: minted-in-text}所示。

\begin{listing}[ht!]
    \begin{minted}{python}
        class ListNode:
            def __init__(self, x):
                self.val = x
                self.next = None
    \end{minted}
    \caption{\mintinline{latex}{minted}环境在文中的显示效果示例}
    \label{lst: minted-in-text}
\end{listing}

您也可能在行内插入代码片段,可以使用 \mintinline{latex}{\mintinline{language}{code}} 命令。例如,\mintinline{latex}{\mintinline{python}{ListNode}} 会生成 \mintinline{python}{ListNode}。
