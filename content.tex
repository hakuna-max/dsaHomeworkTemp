%!TEX root = main.tex
% !TEX encoding = UTF-8
% content.tex
% 请在这里撰写你所有的作业内容

% --- 第一题 ---
\section{递归函数实现:计算各位数字之和}

请实现一个递归函数 \mintinline{python}{sum_of_digits(n: int) -> int},计算一个非负整数的各位数字之和。例如,\mintinline{python}{sum_of_digits(1234)} 应返回 10。

\subsection*{解:}

求解思路:(请在此处描述函数设计思路)

Listing~\ref{lst: sum-of-digits-function} 为该递归函数的实现。

\begin{listing}[ht!]
    \begin{minted}{python}
        def sum_of_digits(n: int) -> int:
            # 请在此处完成函数实现
    \end{minted}
    \caption{递归实现计算各位数字之和的\mintinline{python}{sum_of_digits}函数示例}
    \label{lst: sum-of-digits-function}
\end{listing}

\section{递归函数实现:判断回文字符串}

请实现一个递归函数 \mintinline{python}{is_palindrome(s: str) -> bool},判断一个字符串是否是回文字符串(即正读和反读都一样)。例如,\mintinline{python}{is_palindrome("racecar")} 应返回 \mintinline{python}{True},而 \mintinline{python}{is_palindrome("hello")} 应返回 \mintinline{python}{False}。

\subsection*{解:}

求解思路:(请在此处描述函数设计思路)

Listing~\ref{lst: is_palindrome} 为该递归函数的实现。

\begin{listing}[ht!]
    \begin{minted}{python}
        def is_palindrome(s: str) -> bool:
            # 请在此处完成函数实现
    \end{minted}
    \caption{递归实现判断回文字符串的\mintinline{python}{is_palindrome}函数示例}
    \label{lst: is_palindrome}
\end{listing}

\section{递归函数实现:二分查找}

请实现一个递归函数 \mintinline{python}{binary_search(arr: List[int], target: int) -> int},在一个已排序的整数数组中查找目标值。如果找到,返回其索引;否则,返回 -1。要求使用二分查找算法。可以假设数组中没有重复元素。

\subsection*{解:}

求解思路:(请在此处描述函数设计思路)

Listing~\ref{lst: binary_search} 为该递归函数的实现。

\begin{listing}[ht!]
    \begin{minted}{python}
        def binary_search(arr: List[int], target: int) -> int:
            # 请在此处完成函数实现
    \end{minted}
    \caption{递归实现判断回文字符串的\mintinline{python}{binary_search}函数示例}
    \label{lst: binary_search}
\end{listing}

\section{递归函数实现:计算两个正整数的和}

下面给出了两个不同的过程(procedure),它们都定义了如何计算两个正整数的和。这两个过程都构建于另外两个更基本的过程之上:
\begin{itemize}
    \item \mintinline{python}{inc}:该过程会将其参数的值增加 1。\mintinline{python}{inc = lambda x: x + 1}。
    \item \mintinline{python}{dec}:该过程会将其参数的值减少 1。\mintinline{python}{dec = lambda x: x - 1}。
\end{itemize}
过程定义一:
\begin{minted}{python}
                def add(a, b):
                    if a == 0:
                        return b
                    else:
                        return inc(add(dec(a), b))
            \end{minted}
过程定义二:
\begin{minted}{python}
                def add(a, b):
                    if a == 0:
                        return b
                    else:
                        return add(dec(a), inc(b))    
            \end{minted}
请分析这两个过程的计算过程,回答以下问题:
\begin{enumerate}
    \item 请使用替换模型(substitution model),分别推演和展示当计算 \mintinline{python}{add(4, 5)} 时,上述两个过程所生成的计算过程是怎样的。
    \item 请判断,这两个计算过程哪一个是递归过程(Recursive Process),哪一个是迭代过程(Iterative Process)?并说明理由。
\end{enumerate}


\subsection*{问题1的解:}

(请在此处展示推演过程)


\subsection*{问题2的解:}


