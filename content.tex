%!TEX root = main.tex
% !TEX encoding = UTF-8
% content.tex
% 请在这里撰写你所有的作业内容

\section{}

请在\mintinline{python}{SimplifiedList}基础上,尝试实现 \mintinline{python}{SimplifiedList} 类的末尾删除操作 \mintinline{python}{pop()} 方法。这个方法应该从列表末尾删除一个元素,并返回被删除的元素,同时更新内部状态。你也应该仿照 \mintinline{python}{__getitem__()} 方法,添加边界检查,确保在删除时不会访问到不存在的索引(如索引小于 0 或大于等于当前长度)。如果删除成功,返回被删除的元素,否则抛出 \mintinline{python}{IndexError} 异常。

\subsection*{解:}

\begin{minted}{python}
    import ctypes
    from collections import namedtuple

    # 定义“管理区”
    # 使用 namedtuple 来表示这个管理结构,它包含三个核心元数据:
    # n: 当前存储的元素数量 (size)
    # capacity: 底层数组的总容量
    # data: 指向底层数组的引用 (pointer)
    ListInternal = namedtuple("ListInternal", ["n", "capacity", "data"])


    class SimplifiedList:
        """一个简化版的动态数组。"""

        def __init__(self):
            """初始化一个空的列表。"""
            # 初始时,列表为空,大小和容量都为0。
            # self._internal 完美地模拟了那个“列表对象头”。
            self._internal = ListInternal(0, 0, self._make_array(0))

        def __len__(self):
            """
            获取列表长度,复杂度 O(1)。
            这完美地解释了为什么 len() 如此之快。
            """
            return self._internal.n

        def __getitem__(self, k):
            """
            通过索引获取元素,复杂度 O(1)。
            这是数组的随机访问特性。
            """
            # 边界检查
            if not 0 <= k < self._internal.n:
                raise IndexError("list index out of range")
            return self._internal.data[k]

        def append(self, obj):
            """
            在列表末尾添加元素,摊还复杂度 O(1)。
            """
            # 核心逻辑:如果空间不足,就进行扩容
            if self._internal.n == self._internal.capacity:
                self._resize(
                    max(1, 2 * self._internal.capacity)
                )  # 至少扩容到1,或当前容量的两倍

            # 将新元素放入第一个可用的空位
            self._internal.data[self._internal.n] = obj

            # 更新 size 计数器(因为 namedtuple 不可变,我们创建一个新实例)
            self._internal = self._internal._replace(n=self._internal.n + 1)


        def pop(self):
            """
            从列表末尾删除一个元素并返回它,复杂度 O(1)。
            """
            # 请在此添加您的代码

            # 边界检查


        def _resize(self, new_capacity):
            """
            私有方法,用于扩容底层数组(这是一个 O(n) 操作)。
            """
            # 1. 创建一个新的、更大的底层数组
            new_array = self._make_array(new_capacity)

            # 2. 将旧数组中的所有元素复制到新数组
            for i in range(self._internal.n):
                new_array[i] = self._internal.data[i]

            # 3. 更新内部状态,指向新的数组和容量
            self._internal = ListInternal(self._internal.n, new_capacity, new_array)

        def _make_array(self, capacity):
            """
            创建一个类似C语言的、固定大小的底层数组。
            """
            # (ctypes.py_object * capacity)() 创建一个能存放'capacity'个Python对象的数组
            return (ctypes.py_object * capacity)()

        def __repr__(self):
            """
            提供一个美观的打印输出,类似于真实的 list。
            """
            elements = [repr(self[i]) for i in range(len(self))]
            return f"[{', '.join(elements)}]"
\end{minted}

\section{}

为什么我们说数组支持“随机访问”,而链表只支持“顺序访问”?请从它们在内存中的存储方式来解释其根本原因。

\subsection*{解:}

\section{}

简述单向链表、双向链表和循环链表各自的优缺点,并为每种结构举出一个典型的应用场景。

\subsection*{解:}

\section{}
\textbf{场景分析:}你正在为一个音乐 App 设计“播放队列”功能。用户有以下几种核心需求:
\begin{enumerate}
    \item 随时查看当前播放歌曲在队列中的位置(即获取其索引)。
    \item 支持“下一首”播放(即移动到下一个元素)。
    \item 允许用户将任意一首歌“置顶”(即将任意一个元素移动到队列头部)。
    \item 允许用户删除队列中的任意一首歌。
\end{enumerate}
请分别使用数组和双向链表来分析实现这四个功能的复杂度,并最终给出你的技术选型和理由。建议使用一个表格来清晰地进行对比。
\subsection*{解:}

\section{}
给定一个单向链表的头节点 \mintinline{python}{head},请设计一个算法,要求只遍历一次链表,就能找到链表的倒数第 $k$ 个节点。请描述你的算法思路(提示:可以使用双指针法),并分析其时间复杂度和空间复杂度。

\subsection*{解:}


\section{}
请实现一个双向链表的 \mintinline{python}{insert_before} 方法,用于在指定节点前插入一个新节点。要求时间复杂度为 $O(1)$。
\subsection*{解:}

\begin{minted}{python}
    class Node:
        def __init__(self, data):
            self.data = data
            self.next = None
            self.prev = None

    class DoublyLinkedList:
        def __init__(self):
            self.head = None
            self.tail = None

        def insert_before(self, node, new_node):
            # 请在此添加您的代码
\end{minted}

\section{}
请实现一个双向循环链表(circular doubly linked list),其中每个节点既有 \mintinline{python}{next} 指针指向后继节点,也有 \mintinline{python}{prev} 指针指向前驱节点。要求实现基本的插入、删除和遍历操作。
\subsection*{解:}

\begin{minted}{python}
    class Node:
        def __init__(self, data):
            self.data = data
            self.next = None
            self.prev = None

    class CircularDoublyLinkedList:
        # 请在此添加您的代码
\end{minted}

\section{}
请实现一个双向链表的 \mintinline{python}{reverse} 方法,用于反转链表。要求时间复杂度为 $O(n)$。例如,一个链表 A -> B -> C -> None,反转后应为 C -> B -> A -> None。

\subsection*{解:}

\begin{minted}{python}
    class Node:
        def __init__(self, data):
            self.data = data
            self.next = None
            self.prev = None

    class DoublyLinkedList:
        def __init__(self):
            self.head = None
            self.tail = None

        def reverse(self):
            prev = None
            current = self.head
            while current:
                # 在这里补充你的代码
                ...
            self.head = prev
\end{minted}

\section{}
给定两个已排序的单向链表 \mintinline{python}{list1} 和 \mintinline{python}{list2},请编写一个函数 \mintinline{python}{merge_sorted_lists(list1, list2)},将它们合并成一个新的、仍然有序的链表,并返回新链表的头节点。不允许创建新的节点,只能通过修改原有两个链表的节点指针来完成。

\subsection*{解:}

\begin{minted}{python}
    def merge_sorted_lists(list1, list2):
        # 在这里补充你的代码
\end{minted}

