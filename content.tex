%!TEX root = main.tex
% !TEX encoding = UTF-8
% content.tex
% 请在这里撰写你所有的作业内容

\section{循环结构分析}

请分析以下 Python 代码片段的时间复杂度,并使用大 $O$ 表示法给出结果。请简要说明你的推导过程。

\begin{minted}{python}
    def exercise_one(n):
        total = 0
        # 外层循环
        for i in range(n):
            # 内层循环
            for j in range(n * n):
                total += i * j
        return total
\end{minted}

\subsection*{解:}


\section{连续与条件结构分析 (规则 3 \& 4)}

请分析以下 Python 函数的整体时间复杂度。请分别指出代码中第一部分和第二部分(\mintinline{python}{if}/\mintinline{python}{else} 结构)的复杂度,并根据规则推导出最终的整体复杂度。

\begin{minted}{python}
    def exercise_two(n):
        # 第一部分
        result = 0
        for i in range(n // 2):
            result += i

        # 第二部分
        if n > 100:
            # if 分支
            for i in range(n):
                for j in range(n):
                    result += (i + j)
        else:
            # else 分支
            result += n
        
        return result
\end{minted}

\subsection*{解:}


\section{对数复杂度分析}

与课程中分析的循环不同,有些循环的计数器(\mintinline{python}{counter})并不是线性增加的。请分析以下代码片段的时间复杂度,并解释为什么它的增长趋势与普通循环不同。

\begin{minted}{python}
    def exercise_three(n):
        i = 1
        count = 0
        while i < n:
            i = i * 2  # 关键步骤
            count += 1
        return count
\end{minted}

\subsection*{解:}


\section{主定理直接应用 (规则 5)}

一个递归算法的运行时间由以下递归关系式描述:
\begin{equation}
    T(n) = 8T\left(\frac{n}{2}\right) + O(n^2)
\end{equation}

请根据主定理(Master Theorem)判断该算法的时间复杂度属于哪一种情况,并写出最终的复杂度结果。

\subsection*{解:}


\section{从代码到递归关系式 (规则 5)}

请分析以下递归函数 \mintinline{python}{recursive_func} 的时间复杂度。请先写出描述其运行时间的递归关系式,然后使用主定理求解。
\begin{minted}{python}
    def recursive_func(n):
        if n <= 1:
            return 1
        
        # 递归调用
        result = recursive_func(n / 3) + recursive_func(n / 3) + recursive_func(n / 3)

        # 非递归部分的开销
        for i in range(n):
            result += i
            
        return result
\end{minted}

\subsection*{解:}

\section{算法比较}

假设有两个算法 $A$ 和 $B$ 用于解决同一个问题。算法 $A$ 的时间复杂度为 $O(n^2)$,算法 $B$ 的时间复杂度为 $O(n \log n)$。当处理一个规模为 $n=1,000,000$ 的超大规模数据集时,你会选择哪个算法?为什么?这种选择体现了渐进分析的什么核心思想?

\subsection*{解:}

\section{综合分析}

请综合运用时间复杂度分析的各项规则,分析以下 \mintinline{python}{comprehensive_analysis} 函数的整体时间复杂度,并给出详细的推导过程。

\begin{minted}{python}
    def comprehensive_analysis(n):
        # 第一部分
        for i in range(n):
            for j in range(i):
                # O(1) 操作
                pass

        # 第二部分
        if n % 3 == 0:
            # 递归调用
            def recursive_part(x):
                if x <= 1:
                    return 1
                return 4 * recursive_part(x / 2)
            recursive_part(n)
        else:
            # 简单循环
            for i in range(n):
                # O(1) 操作
                pass
                
        return n
\end{minted}

\subsection*{解:}

\section{扩展:反向思考}

请设计一个 Python 函数,使其时间复杂度恰好为 $O(n\sqrt{n})$。\textbf{提示}:可以考虑使用嵌套循环,其中内层循环的迭代次数与外层循环的变量 \mintinline{python}{i} 相关。

\subsection*{解:}