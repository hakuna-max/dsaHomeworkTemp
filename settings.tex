%!TEX root = main.tex
% !TEX encoding = UTF-8
% settings.tex: LaTeX 设置文件

% --- Document Class ---
% 使用 article 类,字体大小为 12pt
\documentclass[12pt]{article}

% --- Packages ---
% 页面边距设置
\usepackage[a4paper, margin=1in]{geometry}

% 颜色支持
\usepackage[dvipsnames, table, HTML]{xcolor}

% 中文支持
% ===== 设置字体 =====
\usepackage{fontspec} % 加载 fontspec 包支持字体设置
\usepackage{xeCJK} % 加载 xeCJK 包支持中文

% 设置主字体(英文正文)
\setmainfont{IBM Plex Serif}

% 设置无衬线字体
\setsansfont{IBM Plex Sans}

% 设置代码字体
\setmonofont{IBM Plex Mono}
\setCJKmonofont{LXGW WenKai Mono} % 设置 CJK 代码字体

% 设置 CJK 主字体
\setCJKmainfont[AutoFakeBold=true]{LXGW WenKai} % 设置 CJK 主字体
% 设置 CJK 无衬线字体
\setCJKsansfont{IBM Plex Sans SC} % 设置 CJK 无衬线字体

% \usepackage{ctex}

% 标题和章节格式
\usepackage{titlesec}
\usepackage{zhnumber}
\renewcommand{\thesection}{问题\zhnum{section}}
\renewcommand{\thesubsection}{\arabic{subsection}}

\titleformat{\section}
{\normalfont\bfseries}
{\thesection:}
{0em}
{}

% 设置 subsection 格式(可选)
\titleformat{\subsection}
{\normalfont\bfseries}
{(\thesubsection)}
{0.5em}
{}

% 数学公式支持
\usepackage{amsmath}
\usepackage{amssymb}
\usepackage{amsfonts}

% 算法伪代码支持
\usepackage[ruled,vlined]{algorithm2e} % 如涉及算法伪代码,可启用此宏包

% 插入代码块支持
% minted 宏包提供了更美观的代码高亮,需要 Python 和 Pygments 库支持
\usepackage{minted}
\setminted{
    linenos,
    xleftmargin=2em,
    autogobble=true,
    numbers=left,
    numbersep=1em,
    baselinestretch=1.2,
    breaklines=true,
    breakanywhere=true,  % 允许在任何位置断行
    framesep=2mm,  % 边框间距
    % fontsize=\small,
    highlightcolor=OliveGreen!10,
    escapeinside=||,  % 可选:允许在代码中添加 LaTeX 命令
}
% mintinline的字体设置,使其与正文字号保持一致
\usepackage{inputenc}
\makeatletter
\newcommand{\currentfontsize}{\fontsize{\f@size}{\f@baselineskip}\selectfont}
\makeatother
\setmintedinline{fontsize=\currentfontsize}

% \usemintedstyle{tango} % 设置默认 Python 风格
\usepackage{caption}  % 增强的标题功能
\usepackage{listings}
\lstset{
    basicstyle=\ttfamily\small,
    breaklines=true,
    breakatwhitespace=true,
    columns=flexible,
}

% 自定义 listing 环境的标题格式
\captionsetup[listing]{
    format=hang,
    font=small,
    labelfont=bf,
    textfont=normalfont,
    singlelinecheck=false,
    margin=0pt,
    skip=5pt
}


% 插入图片支持
\usepackage{graphicx}
\graphicspath{{images/}} % 图片可以统一放在 images 目录下

% 标题和章节格式
\usepackage{titlesec}
\titleformat{\section}
{\normalfont\Large\bfseries}
{\thesection}
{1em}
{}

% 页眉页脚设置
\usepackage{fancyhdr}
\setlength{\headheight}{14.5pt} % 增加页眉高度以避免警告
\pagestyle{fancy}
\fancyhf{} % 清空页眉页脚
\lhead{\courseName(\semester)}
\rhead{\studentName \quad \studentID}
\chead{\homeworkTitle}
\cfoot{\thepage}

% 行间距设置
\usepackage{setspace}
\setstretch{1.5} % 行间距设置为1.5倍

% --- Custom Commands ---
\newcommand{\courseName}{数据结构与算法} % 课程名称
\newcommand{\semester}{\small 2025-2026-1}
\newcommand{\homeworkTitle}{作业3:数组与链表} % 作业标题
% \newcommand{\studentName}{张露平} % 请修改为你的姓名
% \newcommand{\studentID}{20190092} % 请修改为你的学号
% \newcommand{\profName}{指导教师:张露平}
\newcommand{\submissionDate}{\today} % 提交日期,\today 会自动生成当天日期

% --- Title ---
% 定义作业标题格式
\title{
    \centering
    \vspace{-2cm}
    \textbf{\courseName} \\
    \large \homeworkTitle
}
\author{
    \centering
    \studentName \quad \studentID \\
    % \profName
}
\date{
    \centering
    \submissionDate
}

\makeatletter
\renewcommand{\maketitle}{
    \begin{center}
        {\LARGE \@title \par}
        \vspace{1em}
        {\large \@author \par}
        \vspace{0.5em}
        {\large \@date \par}
    \end{center}
    \vspace{2em}
}
\makeatother

% --- End of settings.tex ---